\documentclass[11pt, oneside]{article}    

\usepackage{geometry}
\geometry{letterpaper}                          
\usepackage{graphicx}
\usepackage{color}
\usepackage{xcolor}
\usepackage{amssymb}
\usepackage{hyperref}
\title{NeurIPS  2019 Competition proposal:}
\author{Organizer1\thanks{The Leader organizer should be the first author of the proposal.} \and Organizer2 \and Organizer2 \and ... \and Organizern \\ \and
{\tt contact-email@somewhere.org}\\
}
\date{\today}

\begin{document}
\maketitle

\begin{abstract}
Briefly describe your competition, either a ``regular competition'' running over a few months before the challenge or a live/demonstration competition (competition of demonstrations or live contest) running at the conference site.
Summarize the background, available data, methods, available baseline, and potential impact.
\end{abstract}

\subsection*{Keywords}
Up to five, from generic to specific.
\subsection*{Competition type} Live, regular, or both.


\section{Competition description}

\subsection{Background and impact}

Provide some background on the problem approached by the competition and fields of research involved. Describe the scope and indicate the anticipated impact of the competition prepared (economical, humanitarian, societal, etc.). \textbf{Please note that tasks of humanitarian and/or positive societal impact will be particularly considered this year.}


Justify the relevance of the problem to the targeted by the NeurIPS community and indicate whether it is of interest to a large audience or limited to a small number of domain experts (estimate the number of participants). A good consequence for a competition is to learn something new by answering a scientific question or make a significant technical advance.

Describe typical real life scenarios and/or delivery vehicles for the competition. This is particularly important for live competitions, but may also be relevant to regular competitions. For instance: what is the application setting, will you use a virtual or a game environment, what situation(s)/context(s) will participants/players/agents be facing?

Put special emphasis on relating the, necessarily simplified, task of the competition to a real problem faced in industry or academia. If the task cannot be cast in those terms, provide a detailed hypothetical scenario and focus on relevance to NeurIPS.


\subsection{Novelty}

Have you heard about similar competitions in the past? If yes, describe the key differences.
Indicate whether this is a completely new competition, a competition part of a series, eventually re-using old data.

\subsection{Data}

If the competition uses an evaluation based on the analysis of data,
please provide detailed information about the available data and
their annotations, as well as permissions or licenses to use such data.

If new data are collected or generated, provide details on the procedure, including permissions to collect such data obtained by an ethics committee, if human subjects are involved. In this case, it must be clear in the document that the data will be ready prior to the official launch of the
competition. 

Please justify that: (1) you have access to large
enough datasets to make the competition interesting and draw
conclusive
 results; (2) the data will be made freely available;(3) the ground truth has been kept confidential.

\subsection{Tasks and application scenarios}

Describe the tasks of the competition and explain to which specific real-world scenario(s) they correspond to. If the competition does not lend itself
to real-world scenarios, provide a justification. Justify that the problem posed are scientifically or technically challenging but not impossible to
solve. If data are used, think of illustrating the same scientific problem using several datasets from various application domains.


\subsection{Metrics}

For quantitative evaluations, select a scoring metric and justify
that it effectively assesses the efficacy of solving the problem
at hand. It should be possible to evaluate the results
objectively. If no metrics are used, explain how the evaluation
will be carried out. Explain how error bars will be computed and/or how the significance in performance difference between participants will be evaluated.

You can include subjective measures provided by human judges (particularly for live /  demonstration competitions). In that case, describe the judging criteria, which must be as orthogonal as possible, sensible, and specific. Provide details on the judging protocol, especially how to break ties between judges. Explain how judges will be recruited and, if possible, give a tentative list of judges, justifying their qualifications.  

\subsection{Baselines, code, and material provided}

Specify what are (will be) the baselines for the competition. Provide preliminary results, if available.

Indicate
whether there will be available code for the participants to get started with (``starting kit''). For certain competitions, material provided may include a hardware platform.

\subsection{Tutorial and documentation}

Provide a reference to a white paper you wrote describing the
problem and/or explain what tutorial material you will provide.


\section{Organizational aspects}
\subsection{Protocol}

Explain the procedure of the competition: what the participants will have to do, what will be submitted (results or code), and the evaluation procedure.
Will there be several phases? Will you use a competition platform with on-line submissions and a leader board? Indicate means of preventing cheating.
Provide your plan to organize beta tests of your protocol and/or platform.


\subsection{Rules}

In this section, please provide:

\begin{itemize}
	\item A verbatim copy of (a draft of) the contest rules given to the contestants. 
	\item A discussion of those rules and how they lead to the desired outcome of your competition. 
	\item A discussion about cheating prevention.
\end{itemize}

Choose inclusive rules, which allow the broadest possible participation from the NeurIPS audience. 

\subsection{Schedule and readiness}

Provide a time line for competition preparation and for running the competition itself. Propose a reasonable schedule leaving enough time for the organizers
to prepare the event (a few months), enough time for the participants to develop their methods (e.g. 90 days), enough time for the organizers to review the entries, analyze and publish the results. 

For live/demonstration competitions, indicate how much overall time you will need (we do not guarantee all competitions will get the time they request). Also provide a detailed schedule for the on-site contest at NeurIPS. This schedule should at least include times for introduction talks/video presentations, demos by the contestants, and an award ceremony. 

 Will the participants need to prepare their contribution in advance (e.g. prepare a demonstration) and bring ready-made software and hardware to the competition site? Or, on the contrary, can will they be provided with everything they need to enter the competition on the day of the competition? Do they need to register in advance? What can they expect to be available to them on the premises of the live competition (tables, outlets, hardware, software and network connectivity)? What do they need to bring (multiple connectors, extension cords, etc.)?


Indicate what, at the time of writing this proposal, is already ready.


\subsection{Competition promotion}

Describe the plan that organizers have to promote participation in the competition (e.g., mailing lists in which the call will be distributed, invited talks, etc.).  


Please also describe your plan  for attracting participants of groups under-represented at NeurIPS. 





\section{Resources}

\subsection{Organizing team}

Provide a short biography of all team members, stressing their competence for their assignments in the competition organization. Please note that diversity in the organizing team is encouraged, please elaborate on this aspect as well.  Make sure to include: coordinators, data providers, platform administrators, baseline method providers, beta testers, and evaluators. 
\subsection{Resources provided by organizers, including prizes}

Describe your resources (computers, support staff, equipment, sponsors, and available prizes and travel awards).

For live/demonstration competitions, explain how much will be provided by the organizers (demo framework, software, hardware) and what the participants will need to contribute (laptop, phone, other hardware or software).

\subsection{Support and facilities requested}
Please indicate the kind of support and facilities you need from the conference. 

For live/demonstration competitions, indicate what you will need on site from NeurIPS in order to run the live competition (size of room, duration, tables, outlets, network connectivity). We do not commit to provide all such support free-of-charge. 



\end{document}
