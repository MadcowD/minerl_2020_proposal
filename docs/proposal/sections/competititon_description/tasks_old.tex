% %!TEX root = ../../main.tex

% \subsection{Tasks and application scenarios}

% % wanted to make it more clear that we expect participants to not obtain diamond in the first go, rather to obtain lesser milestones early on

%     For the purpose of this competition we define ... 
%     \todo[inline]{define sub-task, task}
%      a task to be a tuple of inital conditions and final conditions in Minecraft
%      a sub-task is 

%     One of the main features of this competition is that the task selected allows agents to exploit human demonstrations offered by the MineRLv0 dataset to learn policies efficiently. 

%     The purpose of this competition is to advance the state-of-the-art in sample-efficient reinforcement learning and make these techniques accessible to the broader community. Therefore, the competition is structured differently from prior reinforcement learning competitions at NeurIPS.

%     Participants will be challenged to build learning algorithms that efficiently achieve exceptional performance in a hierarchical, sparse, and extended time-horizon environment in a training window of only 4 days and a maximum of 20 million environment samples. This training window ensures techniques are comparable and will be accessible to all levels of participants.
    
%     In order to make the challenge tractable, participants will be provided a large number of human demonstrations annotated by subtask 
%     and allowed to freely learn from these samples to achieve maximum performance. Further, given the difficulty of the task, participants will receive recognition for completing specific subtasks in the environment. (N = 2, S = 20 million)

%     \subsubsection{Task definition: \texttt{Obtain Diamond}}
%     Agents begin without resources in a randomized map location and must find and mine a diamond ore block within the world. In Minecraft, diamond ore is rare (0.0056\% of blocks) and requires third tier tools (iron\_pickaxe) to mine. This forms a natural hierarchical task requiring the agent to craft three successive tools each requiring the preceding tier as well as explore safely while searching for the ore block. See figure \ref{fig:hierarchicality} for an example of human sub-task orderings for \texttt{obtain diamond}.

%         %When mining using the most efficient tool in the game it takes 9 million samples on average to find and mine a diamond ore. Despite this over half of sampled players were able to discover and recover a diamond in less than 15 minutes.   

%     To avoid hand-engineering policies and strategies which exploit the environment, as has plagued other reinforcement learning competitions in the past, participants will develop their learning algorithms on environments and datasets whose textures, lighting conditions, colors, and geometries are different than those on which their algorithms will be tested and evaluated. Further, participants will be given two instances of the environment and dataset with the foregoing perturbations, and these environments and datasets will serve as ``development'' and ``validation'' sets for the participants learning procedures.


%     Application scenarios:
%     \subsubsection{Application scenarios}
%     \todo[inline]{todo}


%     % ; in particular, success in sample efficient reinforcement learning not only depends on the raw performance of participants’ agents but also the manner in which they are trained. Thus, the competition task is as follows.

%     % One of the mine fratures of the compitition is the tight integration with human demonstrations and hierearcical tasks. The main task praticipants are asked to solve relies on deeply hierarcical policies such as safe navigation, resource collection, mineing, and tool-making.  

%     % Our task - obtain diamond - requires long-term planning and

%     %thoughts - should we allow participants a different gamma setting, fov, or resolution

        
