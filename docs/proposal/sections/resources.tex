\subsection{Organizing team}

\subsubsection{Organizers}

\paragraph{William H. Guss.} William Guss is a research scientist at OpenAI and Ph.D. student in the Machine Learning Department at CMU. William co-created the \minenet{} dataset and lead the MineRL  competition at NeurIPS 2019. He is advised by Dr. Ruslan Salakhutdinov
and his research spans sample-efficient reinforcement learning and deep learning theory.  William completed his bachelors in Pure Mathematics at UC Berkeley where he was awarded the Regents' and Chancellor's Scholarship, the highest honor awarded to incoming undergraduates. During his time at Berkeley, William received the Amazon Alexa Prize Grant for the development of conversational AI and co-founded Machine Learning at Berkeley. William is from Salt Lake City, Utah and grew up in an economically impacted, low-income neighborhood without basic access to computational resources. As a result, William is committed to working towards developing research and initiatives which promote socioeconomically-equal access to AI/ML systems and their development.

\paragraph{Mario Ynocente Castro.} Mario is an Engineer at Preferred Networks. In 2017, he received a Masters in Applied Mathematics at École polytechnique and a Masters in Machine Learning at École Normal Supérieure de Paris-Saclay. His current work focuses on applications of Reinforcement Learning and Imitation Learning.

\paragraph{Sam Devlin.} Sam Devlin is a Senior Researcher in the Game Intelligence and Reinforcement Learning research groups at Microsoft Research, Cambridge (UK). He received his PhD on multi-agent reinforcement learning in 2013 from the University of York. Sam has previously co-organised the Text-Based Adventure AI Competition in 2016 \& 2017 and the Multi-Agent Reinforcement Learning in Minecraft (MARLO) Competition in 2018.

\paragraph{Brandon Houghton.} Brandon Houghton is a Machine Learning Engineer at OpenAI and co-creator of the \minenet{} dataset. Graduating from the School of Computer Science at Carnegie Mellon University, Brandon's work focuses on developing techniques to enable agents to interact with the real world through virtual sandbox worlds such as Minecraft. He has worked on many machine learning projects, such as discovering model invariants in physical systems as well as learning lane boundaries for autonomous driving. 

\paragraph{Noboru Sean Kuno.} Noboru Sean Kuno is a Senior Research Program Manager at Microsoft Research in Redmond, USA. He is a member of Artificial Intelligence Engaged team of Microsoft Research Outreach. He leads the design, launch and development of research programs for AI projects such as Project Malmo, working in partnership with research communities and universities worldwide.

\paragraph{Crissman Loomis.} Crissman works for Preferred Networks, a Japanese AI startup that applies the latest deep machine learning algorithms to industrial applications, like self-driving cars, factory automation, or medicine development. At Preferred Networks, he has supported the development and adoption of open source frameworks, including the deep learning framework Chainer and more recently the hyperparameter optimization library Optuna.

\paragraph{Stephanie Milani.} Stephanie Milani is a Ph.D. student in the Machine Learning Department at Carnegie Mellon University.
She is advised by Dr. Fei Fang and her research interests include sequential decision-making problems, with an emphasis on reinforcement learning.
In 2019, she completed her B.S. in Computer Science and her B.A. in Psychology at the University of Maryland, Baltimore County, and she co-organized the 2019 MineRL competition..
Since 2016, she has worked to increase the participation of underrepresented groups in CS and AI at the local and state level. 
For these efforts, she has been nationally recognized 
through a Newman Civic Fellowship. 

\paragraph{Sharada Mohanty.} Sharada Mohanty is the CEO and Co-founder of AIcrowd, an open-source platform encouraging reproducible artificial intelligence research. 
He was the co-organizer of many large-scale machine learning competitions, such as NeurIPS 2017: Learning to Run Challenge, NeurIPS 2018: AI for Prosthetics Challenge, NeurIPS 2018: Adversarial Vision Challenge, NeurIPS 2019 : MineRL Competition, NeurIPS 2019: Disentanglement Challenge, NeurIPS 2019: REAL Robots Challenge. 
During his Ph.D. at EPFL, he worked on numerous problems at the intersection of AI and health, with a strong interest in reinforcement learning.  
In his current role, he focuses on building better engineering tools for AI researchers and making research in AI accessible to a larger community of engineers. 

\paragraph{Keisuke Nakata.} Keisuke Nakata is a machine learning engineer at Preferred Networks, Inc. He mainly works on machine learning applications in real-world industry settings. Particularly, his interests lie in creating reinforcement learning algorithms and frameworks.

\paragraph{Ruslan Salakhutdinov.} Ruslan Salakhutdinov received his Ph.D. in machine learning (computer science) from the University of Toronto in 2009. After spending two post-doctoral years at the Massachusetts Institute of Technology Artificial Intelligence Lab, he joined the University of Toronto as an Assistant Professor in the Department of Computer Science and Department of Statistics. In February of 2016, he joined the Machine Learning Department at Carnegie Mellon University as an Associate Professor. Ruslan's primary interests lie in deep learning, machine learning, and large-scale optimization. His main research goal is to understand the computational and statistical principles required for discovering structure in large amounts of data. He is an action editor of the Journal of Machine Learning Research and served on the senior programme committee of several learning conferences including NeurIPS and ICML. He is an Alfred P. Sloan Research Fellow, Microsoft Research Faculty Fellow, Canada Research Chair in Statistical Machine Learning, a recipient of the Early Researcher Award, Connaught New Researcher Award, Google Faculty Award, Nvidia's Pioneers of AI award, and is a Senior Fellow of the Canadian Institute for Advanced Research.

\paragraph{John Schulman.} John Schulman is a researcher and founding member of OpenAI, where he leads the reinforcement learning team. He received a PhD from UC Berkeley in 2016, advised by Pieter Abbeel. He was named one of MIT Tech Review’s 35 Innovators Under 35 in 2016.

\paragraph{Shinya Shiroshita.} 
Shinya Shiroshita works for Preferred Networks as an engineer. He graduated from the University of Tokyo, where he majored in computer science. His hobbies are competitive programming and playing board games. In Minecraft, he likes exploring interesting structures and biomes.

\paragraph{Nicholay Topin.} Nicholay Topin is a Machine Learning Ph.D. student advised by Dr. Manuela Veloso at Carnegie Mellon University. His current research focus is explainable deep reinforcement learning systems. Previously, he has worked on knowledge transfer for reinforcement learning and learning acceleration for deep learning architectures. 

\paragraph{Avinash Ummadisingu.} Avinash Ummadisingu works at Preferred Networks on Deep Reinforcement Learning for Robotic Manipulation and the open-source library PFRL (formerly ChainerRL). His areas of interests include building sample efficient reinforcement learning systems and multi-task learning. Prior to that, he was a student at USI, Lugano under the supervision of Prof. Jürgen Schmidhuber and Dr. Paulo E. Rauber of the Swiss AI Lab IDSIA.

\paragraph{Oriol Vinyals.} Oriol Vinyals is a Principal Scientist at Google DeepMind, and a team lead of the Deep Learning group. His work focuses on Deep Learning and Artificial Intelligence. Prior to joining DeepMind, Oriol was part of the Google Brain team. He holds a Ph.D. in EECS from the University of California, Berkeley and is a recipient of the 2016 MIT TR35 innovator award. His research has been featured multiple times at the New York Times, Financial Times, WIRED, BBC, etc., and his articles have been cited over 65000 times. His academic involvement includes program chair for the International Conference on Learning Representations (ICLR) of 2017, and 2018. He has also been an area chair for many editions of the NIPS and ICML conferences. Some of his contributions such as seq2seq, knowledge distillation, or TensorFlow are used in Google Translate, Text-To-Speech, and Speech recognition, serving billions of queries every day, and he was the lead researcher of the AlphaStar project, creating an agent that defeated a top professional at the game of StarCraft, achieving Grandmaster level, also featured as the cover of Nature. At DeepMind he continues working on his areas of interest, which include artificial intelligence, with particular emphasis on machine learning, deep learning and reinforcement learning.

\subsubsection{Advisors}
\paragraph{Anca Dragan.} Anca Dragan is an Assistant Professor in the EECS Department at UC Berkeley. Her goal is to enable robots to work with, around, and in support of people. She runs the InterACT Lab, where the focus is on algorithms for human-robot interaction -- algorithms that move beyond the robot's function in isolation, and generate robot behavior that also accounts for interaction and coordination with end-users. The lab works across different applications, from assistive robots, to manufacturing, to autonomous cars, and draw from optimal control, planning, estimation, learning, and cognitive science. She also helped found and serve on the steering committee for the Berkeley AI Research (BAIR) Lab, and am a co-PI of the Center for Human-Compatible AI. She was also honored by the Sloan Fellowship, MIT TR35, the Okawa award, and an NSF CAREER award.

\paragraph{Fei Fang.} Fei Fang is an Assistant Professor at the Institute for Software Research in the School of Computer Science at Carnegie Mellon University. Before joining CMU, she was a Postdoctoral Fellow at the Center for Research on Computation and Society (CRCS) at Harvard University. She received her Ph.D. from the Department of Computer Science at the University of Southern California in June 2016.
Her research lies in the field of artificial intelligence and multi-agent systems, focusing on integrating machine learning with game theory. Her work has been motivated by and applied to security, sustainability, and mobility domains, contributing to the theme of AI for Social Good. 


\paragraph{Chelsea Finn.} Chelsea Finn is an Assistant Professor in Computer Science and Electrical Engineering at Stanford University. Finn's research interests lie in the capability of robots and other agents to develop broadly intelligent behavior through learning and interaction. To this end, her work has included deep learning algorithms for concurrently learning visual perception and control in robotic manipulation skills, inverse reinforcement methods for scalable acquisition of nonlinear reward functions, and meta-learning algorithms that can enable fast, few-shot adaptation in both visual perception and deep reinforcement learning. Finn received her Bachelor's degree in Electrical Engineering and Computer Science at MIT and her PhD in Computer Science at UC Berkeley. Her research has been recognized through the ACM doctoral dissertation award, an NSF graduate fellowship, a Facebook fellowship, the C.V. Ramamoorthy Distinguished Research Award, and the MIT Technology Review 35 under 35 Award, and her work has been covered by various media outlets, including the New York Times, Wired, and Bloomberg. Throughout her career, she has sought to increase the representation of underrepresented minorities within CS and AI by developing an AI outreach camp at Berkeley for underprivileged high school students, a mentoring program for underrepresented undergraduates across four universities, and leading efforts within the WiML and Berkeley WiCSE communities of women researchers.

\paragraph{David Ha.} David Ha 
is a Research Scientist at Google Brain. His research interests include Recurrent Neural Networks, Creative AI, and Evolutionary Computing. Prior to joining Google, He worked at Goldman Sachs as a Managing Director, where he co-ran the fixed-income trading business in Japan. He obtained undergraduate and graduate degrees in Engineering Science and Applied Math from the University of Toronto. 
 
\paragraph{Sergey Levine.} Sergey Levine received a BS and MS in Computer Science from Stanford University in 2009, and a Ph.D. in Computer Science from Stanford University in 2014. He joined the faculty of the Department of Electrical Engineering and Computer Sciences at UC Berkeley in fall 2016. His work focuses on machine learning for decision making and control, with an emphasis on deep learning and reinforcement learning algorithms. Applications of his work include autonomous robots and vehicles, as well as computer vision and graphics. He has previously served as the general chair for the Conference on Robot Learning, program co-chair for the International Conference on Learning Representations, and organizer for numerous workshops at ICML, NeurIPS, and RSS. He has also served as co-organizer on the \emph{Learning to Run} and \emph{AI for Prosthetics} NeurIPS competitions.

\paragraph{Zachary Chase Lipton.} 
Zachary Chase Lipton is an assistant professor of Operations Research and Machine Learning at Carnegie Mellon University. His research spans core machine learning methods and their social impact and addresses diverse application areas, including clinical medicine and natural language processing. Current research focuses include robustness under distribution shift, breast cancer screening, the effective and equitable allocation of organs, and the intersection of causal thinking and the messy high-dimensional data that characterizes modern deep learning applications. He is the founder of the Approximately Correct blog (approximatelycorrect.com) and a founder and co-author of Dive Into Deep Learning, an interactive open-source book drafted entirely through Jupyter notebooks.


\paragraph{Manuela Veloso.} Manuela Veloso is a Herbert A. Simon University Professor at Carnegie Mellon University and the head of AI research at JPMorgan Chase.
She received her Ph.D. in computer science from Carnegie Mellon University in 1992.
Since then, she has been a faculty member at the Carnegie Mellon School of Computer Science.
Her research focuses on artificial intelligence and robotics, across a range of planning, execution, and learning algorithms.
She cofounded the RoboCup Federation and served as president of AAAI from 2011 to 2016.
She is a AAAI, IEEE, AAAS, and ACM fellow.

\subsubsection{Partners and Sponsors}

We are currently in conversation with potential partners for this year's competition.
Last year, we partnered with and/or received support from Microsoft Research, Preferred Networks, NVIDIA, and Artificial Intelligence Journal (AIJ).

\subsection{Resources provided by organizers, including prizes}
{
    \paragraph{Mentorship.}
    We will facilitate a community forum through our publicly available Discord server to enable participants to ask questions, provide feedback, and engage meaningfully with our organizers and advisory board. We hope to foster an active community to collaborate on these hard problems.

    \paragraph{Computing Resources.}
    In concert with our efforts to provide open, democratized access to AI, we are in conversation with potential sponsors to provide compute grants for teams that self identify as lacking access to the necessary compute power to participate in the competition, as we did in the last iteration of the competition.
    We will also provide groups with the evaluation resources for their experiments in Round 2, as we did in the last iteration of the competition.


    \paragraph{Travel Grants and Scholarships.} 
    The competition organizers are committed to increasing the participation of groups traditionally underrepresented in reinforcement learning and, more generally, in machine learning (including, but not limited to: women, LGBTQ individuals, underrepresented racial and ethnic groups, and individuals with disabilities). 
    To that end, we will offer Inclusion@NeurIPS scholarships/travel grants for Round 1 participants who are traditionally underrepresented at NeurIPS to attend the conference. 
    These individuals will be able to apply online for these grants; their applications will be evaluated by the competition organizers and partner affinity groups.
    We also plan to provide travel grants to enable all of the top participants from Round 2 to attend our NeurIPS workshop.
    We are in conversation with potential sponsors about providing funding for these travel grants.

    \paragraph{Prizes.}
    We are currently in discussion about prizes with potential sponsors and/or partners. In the previous competition, we offered 5 NVIDIA GPUs and 10 NVIDIA Jetsons to the top teams. In addition, we provided two prizes for notable research contributions.
}
\subsection{Support and facilities requested}
Due to the quality of sponsorships and industry partnerships secured last year, we only request facility resources and ticket reservations. 
We aim to present at the NeurIPS 2020 Competition Workshop.
We will invite guest speakers, organizers, Round 2 participants, and some Round 1 participants.
To allow these people to attend NeurIPS, we request 30 reservations for NeurIPS. 
We plan to provide funding for teams to travel to the competition.
The organizers will be present at their own expense.


\newpage 